\documentclass[10pt,b5paper, french]{book}

\usepackage{hyperref}
\usepackage{babel}
\usepackage[utf8]{inputenc}
\usepackage{graphicx}
\usepackage{amsmath}
\usepackage{blindtext}
\usepackage{scrextend}
\usepackage{amsthm}
\usepackage{makeidx}
\usepackage{tikz}

\author{Alexandre Sallinen}
\title{Langage et Algorithmes}

\newtheorem{math_def}{Definition}

\makeindex

\begin{document}
\maketitle	

\tableofcontents

\chapter{Théorie des Foncteurs}

\section{Méthode d'abstraction}
\subsubsection{Généralités}

Le principe de généralisation nous permet de raisonner sur des problèmes nouveaux à partir de nos expériences passée. L'abstraction désigne ce principe de soustraction des propriétés spécifiques d'une situation pour en tirer un "squelette" applicable aux cas similaires. Au cours de notre vie nous construisons de nombreux modèles d'abstraction, souvent disjoints les uns des autres, qui permettent de catégoriser nos pensées. Il serait intéréssant de fournir un cadre global d'abstraction permettant, en se spécialisant, de représenter n'importe quel concept concret. Cependant rien ne garantit l'existence d'un tel système, peut-être que certains concepts sont irrémédiablement dissociés, néanmoins certains éléments pousserons à penser qu'un tel système existe. 

\subsubsection{Modèle d'abstraction unique}

Premièrement on ne constate pas d'incompatibilité majeure dans notre monde commun, les généralisation intuitives suffisant dans la grande majorité des problèmes rencontrés. Or si deux systèmes d'abstractions différents existaient, on pourrait imaginer que nous ayons à considérer deux manières de pensée differente face à de nombreux objets.\
Un autre argument serait l'inexorable efficacité des mathématiques à offrir un modèle de pensées. Or les différentes branches des mathématiques sont compatibles entre elle et peuvent être elles-même abstraites. La capacité des mathématiques à modéliser tant de facettes de notre monde provient peut-être du fait que les mathématiques eux mêmes ne sont qu'une formalisation de nos pensées, quoi qu'il en soit, cette déraisonnable efficacité est une piste vers un modèle d'abstraction unique.

\subsubsection{Modèle monoconceptuel}

Cet chapitre se vante de proposer un système d'abstraction complet dans le sens où il peut se spécialiser vers n'importe quel concept existant ou non. De tels systèmes existent déja, (par exemple, la théorie des catégories) cependant, tous ces systèmes incorporent la notion fondemmentale "d'objet", entité existant par elle même mise en relation avec d'autres, et de fonctions, permettant de transformer ces objets. Dans ce chapitre, nous introduirons des arguments intuitifs puis formels visant justifier la construction d'une théorie dénuée du concept "d'objet". Car le concept d'objet à travers son opacité est  différent de celui "d'action" ou de "fonction". Cette dichotomie, motivée par l'idée que les "actions" s'effectuent sur des "choses" suffit pour la plupart des cas. Cependant, la notion d'objet demande évidemment une définition et c'est la que ce modèle trouve ses failles. Cette discussion sera poursuivie dans la \hyperref[sec:Concepts]{sections suivante}, plus adaptée. 
 
\newpage
\subsection{Concepts}
\label{sec:Concepts}

\subsubsection{Sur la nature du canapé}

La nature d'un objet, son essence, fut et est le sujet de longues dissertartions des plus grands esprits de notre monde. Et en cause, une inhabilité à définir un objet par lui même totalement. En plus viennent s'ajouter de nombreux paradoxes, (navire de thésée...) car les définitions floues entraînent un manque de précision face à ce qui \textit{est} et ce qui \textit{n'est pas}. Ces débats trouvent leur place de manière très concrètes dans nos sociétés, de la question des états de conscience ammoindris à celui de la restauration d'une oeuvre d'art, qu'est ce qui définit véritablement, intresèquement quelque chose ? Cette réponse serait intéressante puisque pour abstraire les objets face a nous il va falloir en tirer des propriétés généralisables. Face à cette question indécidable, tout pousse à pe,nser que le problème est peut-être posé de la mauvaise manière. En effet, ces paradoxe ne nous ont jamais empêchés de vivre, d'utiliser ces objets ou d'interagir. Car en effet si la notion de canapé reste opaque, il est facile de décrire ce dernier par les interactions possibles, passées et présentes qui sont possible avec. Ainsi, un canapé sera facilement définissable comme large objet où il est possible de s'asseoir. Ces propriétés correspondent à une forme d'interaction de la part du canapé ou de son entourage. De manière plus subtile, l'image du canapé et même sa forme et l'espace qu'il occupe sont tous définit par ses interactions respectivement de la lumière avec nos yeux, de la surface du canapé avec la lumière et de la surface du canapé avec le sol, et l'air environnant. Mais si il est possible de décrire un objet par ses seules intéractions, il y a t il encore bien du sens à disserter sur sa nature ? Peut-être mais pas ici. Interessons nous plutot au concept "d'objet" généralisé aux objets abstraits que l'on nommera concept.

\subsubsection{Concepts comme collections d'action}
Le terme de concept désignera donce une collection d'actions, elle même sujette d'action. De cette manière on pourra distinguer des concepts imbriqués par leur fonctionnalités. Ainsi si le canapé permet de s'asseoir comfortablement, les pieds du canapé le maintienne à une certaine hauteur tout en en faisant partie.

\subsubsection{Concepts comme action}
Il est aussi possible de définir de manière plus puissante car évitant le terme d'ensemble ou de collection la notion \textit{d'action liante} pour se référer à une propriété commune à toutes les actions d'un concept. Il est important de remarqué qu'un concept peut être composé d'un nombre naturel arbitraire d'action, spécifiquement, une seule suffit.
Une définition informelle et intuitive d'un concept ayant été fondée, il s'en suit celle "d'action".


\subsection{Action}
\subsubsection{Compositions de foncteurs}

Le terme d'action a volontairement été utilisé à la place de \textit{fonction} ou \textit{d'application} car ellez transmettent une notion de relation en théorie des ensemble, trop restreinte pour les nécéssité des actions décrites précédemment. Le terme de foncteur sera retenu de par son étymologie proche du mot fonction et par son suffixe indiquant l'idée d'action. Désormais se pose la question de la notion unificatrice derrière le terme d'action ou de foncteur. Il s'agira de la notion de composition au sens large. Premièrement pour donner une intuition de sa signification on pourra dire qu'une composition est possible lorsque l'on dit que n éléments \textit{interagissent entre eux}, en effet quand on considère les n membres de cette interactions comme des foncteurs on comprend que chaqun agit sur l'autre et donc que leur actions sont \textit{combinées}. C'est cette notion de combinaison de foncteurs que doit transmettre le terme de composition. Selon les propriétés de cette composition elle pourra être par exemple unidirectionelle ou bidirectionelle (pensez au cas du doigt qui touche le canapé, la partie "concrète" du canapé agit sur votre doigt autant que votre doigt agit sur le canapé). Selon la manière dont un foncteur est actuellement composé, son comportement change, c'est enfait sa nature même qui change puisqu'il change de propriétés fondementales. (Par exemple le fait d'avancer vers un objet, l'orsqu'il se compose avec le "concret" de l'objet pourra s'appeler "toucher"). Le foncteur peut se définir ainsi comme suit :

\begin{math_def}
\textbf{Foncteurs} : Un foncteur est une collection de propriétés portant sur les règles de composition avec des concepts.
\end{math_def}

Puisque deux règles de compositions portent toujours sur la manière de composer des foncteurs, il apparait que la composition entre deux foncteurs compatibles  décrit un nouveau foncteur possédant certaines propriétés des deux précédents. 
On constatera que le terme de loies de composition ne peut être dissocié de celui de foncteur, pourtant il est plus commode de faire la distinction entre les deux tout du mooins sur un plan sémantique pour éviter les définitions vagues et pour apporter une intuition plus forte au terme "l'action des foncteurs sur les concepts".

\section{Modèle logique}
\subsection{Formalisation}
\subsubsection{Motivations pour un système formel}

Cette théorie pourrait continuer à être énoncée en termes purement naturels, pourtant on appréciera l'aspect concis et universel d'une syntaxe pour décrire ce modèle théorique. Cependant avec une syntaxe viens une logique et une manière de penser propre à la théorie. Les chapitress suivants seront consacrés à une formalisation de la syntaxe utilisée, de sa grammaire et à la construction d'une fondation logique fonctorielle. 
\\
En vérité de nombreux autres arguments pour un énoncé formel existent. Le premier serait la forte connotation des mots dans un language naturel pouvant faire obstacle à la compréhension d'un concept nouveau. L'incapacité à exprimer certaines idées importantes de manière précise et concise est aussi un grand frein à l'utilisation d'un language naturel. Enfin un avantage des languages formels au dela de leur puissance conceptuelle abstraite est leur capacité à être plus facilement prouvables car ne reposant pas sur des notions polysémique. Ainsi, après leur définitions les termes universels pourront être utilisés sans crainte d'êtres mécompris.
\\\\
Avant de se pencher sur la syntaxe en elle même il faudra néanmoins s'interesser au cadre même de raisonnement logique à utiliser. Ainsi les premiers chapitres justifierons les choix quand à la logique utilisée au cours de cet ouvrage. Le but de cette théorie malgré sa volonté d'abstraction est d'exprimer aussi intuitivement que possible les méthodes et éléments mathématiques des siècles précédents. La suite de cette sous section contient une discussion d'ordre philosophique sur la logique et l'importance d'un point de vue constructiviste ou intuitioniste dans la construction de cette théorie. Si cette discussion n'est certes pas vitale à la compréhension de la théorie, elle explique la motivation ancrée dans notre monde qui porte cet ouvrage.
\newpage

\subsection{Interpreteur}
Aborder les mathématiques dans un cadre général, sans s'interesser à son lecteur, est impossible, quoi qu'il arrive, nous ecrivons pour une audiance précise. La manière d'écrire des mathématique varie d'ailleurs grandement si sson interpreteur doit être un humain ou une machine, dans sa syntaxe comme dans sa nature. Il semble donc nécessaire de prendre en compte cet interpreteur puisqu'il définira la syntaxe et la grammaire de notre language.
L'interpreteur est donc un concept immutable qui agit potentiellement sur n'importe quel concept. Les actions possibles de l'interpreteur sont représentés la syntaxe utilisée. Les définitions et théorèmes prouvés par la suite forment une extension de ses foncteurs.	 

\subsection{Vérité fonctorielle}

Dans un premier temps, une discussion sur la notion de \textit{vrai} et \textit{faux} s'impose. Une proposition est dite vraie lorsque l'interpréteur peut effectuer l'action associé au prédicat. Par exemple, pour un interpréteur humain, savoir si le téléphone se trouve à portée de main reviens à demander si il est possible de le toucher. Et tout naturellement la question de savoir si un nombre donné est un relatif revient à vérifier qu'il n'est pas de partie décimale, ou alors d'un point de vue plus abstrait, de se demander si il satisfait les règles qui sont propres aux relatifs.\
De cette manière la vérité marque l'existance d'un foncteur.

\subsection{Interpréteur universel}

 
\printindex

\end{document}